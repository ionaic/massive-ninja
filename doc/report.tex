\documentclass[10pt, twocolumn]{article}

\usepackage[margin=1in, paperheight=11in, paperwidth=8.5in]{geometry}
\usepackage{float}
\usepackage{graphicx}
\usepackage{tikz}

\title{Computational Vision: Example-Based Texture Synthesis}
\author{Matthew McMullan}
\date{December 11, 2012}

\begin{document}
    \maketitle
    \section{Overview}
    % general overview of what the problem is and what we're trying to do
        Our project focused on the creation of large regions of texture from small example textures.  We accomplish this using a method similar to the one described in Parallel Controllable Texture Synthesis, computing our results using a GPU implementation.  The general idea is to scale the image up (without interpolation), apply jitter (randomly shift regions of the texture in the image), and to correct the jittered texture by comparing to neighborhoods of the exemplar.
    \section{Algorithm}
    % high level discussion of the algorithm and our process as well as the
    %   algorithms described in our papers
        As described in \cite{paratext}, there are three main steps to our implementation: upsample, jitter, and correction.  Upsampling takes the example texture and creates a larger version, with the resulting texture storing coordinates to values in the exemplar.  This texture of coordinates is then used to upsample further in later iterations.  To achieve the upsampling, first the size of the texture is doubled, and the current pixel first creates an extra row and column to the right and down to accomodate the additional pixels before propagating its stored coordinate values to create a larger region containing the value.  This, when used with multiple passes, results in % upsampling equations?

        We then apply jitter, using simplex noise.  The generated noise results in a deterministic texture.  This introduces some randomness to reduce the tiled appearence of the texture.

        Before 
    \section{Implementation}
    % lower level implementation details, talk about GLSL passes and shading and such
    % pipeline diagrams
    % simplex noise
    % glsl opengl
    % subpasses
    \section{Results}
    % discussion and analysis of good, bad, and average case examples; presentation of results
        \subsection{`Good' Cases}
        % particularly good cases
        \subsection{`Bad' Cases}
        % particularly bad cases
    \section{Conclusion}
    % conclusions about the project, some final analyses, future work
    
    \bibliographystyle{plain}
    \bibliography{report.bib}
\end{document}                                                                                                                                                                                                                                                                                                                                                                                                                                                                                                                                                                                                                                                                                                                                                                                                                                                                                                                                                                                                                                                                                                                                                                                                                                                                                                                 
